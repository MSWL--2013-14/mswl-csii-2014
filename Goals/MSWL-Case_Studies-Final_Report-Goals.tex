%%%%%%%%%%%%%%%%%%%%%%%%%%%%%%%%%%%%%%%%%
% by Daniel Gamez
% URJC
%%%%%%%%%%%%%%%%%%%%%%%%%%%%%%%%%%%%%%%%%

%----------------------------------------------------------------------------------------
%	PACKAGES AND OTHER DOCUMENT CONFIGURATIONS
%----------------------------------------------------------------------------------------

\documentclass[11pt]{article} % Default font size is 12pt, it can be changed here

\usepackage{geometry} % Page size to A4
\geometry{a4paper} 

\usepackage{graphicx} % Required for including pictures

\usepackage{float} % Allows putting an [H] in \begin{figure} to specify the exact location of the figure
\usepackage{wrapfig} % Allows in-line images such as the example fish picture

\linespread{1.2} % Line spacing

%\setlength\parindent{0pt} % Uncomment to remove all indentation from paragraphs

\graphicspath{{./Pictures/}} % Specifies the directory where pictures are stored

\usepackage{url}

\usepackage{multirow}


\newenvironment{myindentpar}[1]%
 {\begin{list}{}%
         {\setlength{\leftmargin}{#1}}%
         \item[]%
 }
 {\end{list}}

\begin{document}

%----------------------------------------------------------------------------------------
%	TITLE PAGE
%----------------------------------------------------------------------------------------

\begin{titlepage}

\newcommand{\HRule}{\rule{\linewidth}{0.5mm}} % Defines a new command for the horizontal lines, change thickness here

\center % Center everything on the page

\includegraphics{urjc} \\[0.5cm] % Include logo

\textsc{\LARGE \\Universidad Rey Juan Carlos}\\[1cm] % Name of your university/college

\textsc{\Large Master in Libre Software}\\[0.5cm] % Major heading such as course name
\textsc{\large Case Studies}\\[0.5cm] % Minor heading such as course title

\HRule \\[1.5cm]
{ \huge \bfseries How goals are handled by selected F$\ell$OSS projects}\\[0.4cm] % Title of your document
\HRule \\[1.5cm]

\begin{minipage}{0.4\textwidth}
\begin{flushleft} \large
\emph{Author:}\\
Daniel H.\textsc{G\'amez V.} % Your name
\end{flushleft}
\end{minipage}
%~
\begin{minipage}{0.4\textwidth}
\begin{flushright} \large
\emph{Lecturer:} \\
Gregorio \textsc{Robles} % Lecturer's Name
\end{flushright}
\end{minipage}\\[1cm]

{\large \today}\\[1.8cm] % Date, change the \today to a set date if you want to be precise

\textsc CC BY-SA 3.0\\[0.2cm]
\includegraphics[scale=0.5]{license} \\ % Include license
{\small http://creativecommons.org/licenses/by-sa/3.0/legalcode}\\

\vfill % Fill the rest of the page with whitespace

\end{titlepage}


%----------------------------------------------------------------------------------------
%	ABSTRACT
%----------------------------------------------------------------------------------------

\thispagestyle{empty}
{\bf \huge Abstract}

\vspace{10 mm}

The following report is focused on the goals of some remarkable Free Libre Open Source Software (FLOSS) projects, including specific information about them. These are talks held by people involved in FLOSS showing the projects vertically, providing information regarding the history, the goals, the members, the licenses, the related industry, the technologies, the governance, among others.

\newpage

%----------------------------------------------------------------------------------------
%	TABLE OF CONTENTS
%----------------------------------------------------------------------------------------

\tableofcontents % Include a table of contents

\newpage


%----------------------------------------------------------------------------------------
%	INTRODUCTION
%----------------------------------------------------------------------------------------

\section{Introduction}

Here will be analyzed a few conferences held by key people involved in some significant FLOSS projects, some of them with some longevity like Debian, and others more recent like EdgeBSD. The speakers have presented the projects providing a wealth of information on the history, evolution and composition thereof, and the idea has been to address these projects from the same point of view, attacking a particular topic.\\

In general a software project that follows the essential freedoms proposed by the Free Software Foundation\footnote{http://www.gnu.org/philosophy/free-sw.html}, represents a good measure to determine whether the project is free software or not, and the goals are generally consistent with this criterion.\\

The list of talks that were analyzed, also available on the Internet, is as follows:

\begin{itemize}
  \item How the FreeBSD Project Works\\
  		http://www.youtube.com/watch?v=nNkqKdLm1rU
  \item How Open Source Projects Survive Poisonous People\\
  		http://www.youtube.com/watch?v=ZSFDm3UYkeE
  \item SilverStripe CMS\\
  		http://www.youtube.com/watch?v=9hHHfNJAvi8
  \item Camino\\
  		http://www.youtube.com/watch?v=vGTbVz38CNo
  \item Greg Kroah Hartman on the Linux Kernel\\
  		http://www.youtube.com/watch?v=L2SED6sewRw
  \item Mercurial Project\\
  		http://www.youtube.com/watch?v=1sV8Z\_Lmpt4
  \item Stefano Zacchiroli on Debian: 20 Years and counting\\
  		http://www.youtube.com/watch?v=yKr2qExNoLY
  \item The EdgeBSD Project\\
  		http://www.youtube.com/watch?v=\_D\_iaad5rPo
  \item Debian Secrets what I wish I knew before joining Debian\\
  		http://www.youtube.com/watch?v=jyEmjWvoeNo
\end{itemize}

\newpage


%----------------------------------------------------------------------------------------
%	VIDEO TALKS
%----------------------------------------------------------------------------------------

% \section{Video Talks}

%------------------------------------------------

\section{How the FreeBSD Project Works}

This talk was given by Robert Watson from the University of Cambridge Computer Laboratory, and took place on 2007 at the Google TechTalks meeting. As a current kernel developer Watson has been actively involved in the FreeBSD project since 1999. The talk is about the process of writing Open Source in a large FLOSS project.

%------------------------------------------------

\subsection{Brief description of FreeBSD}

FreeBSD is a FLOSS BSD UNIX based operating system developed by Berkeley University of California back in 1993.\\

It is widely used, and it is possible to categorize three kind of consumers for this operating system:

\begin{itemize}
  \item Large organizations offering network and telecommunication services, such as Internet Service Providers
  \item Hardware manufacturers of appliances or products with embedded operating systems, such as computers or firewalls
  \item Personal computers and the way that everybody access to the Internet nowadays, in the sense that software libraries and protocols from FreeBSD are needed for this purpose
\end{itemize}


\subsection{Main goals of FreeBSD}

\begin{itemize}
  \item Maintain an efficient social process of writing software based on the interaction and feedback of the entities involved
  
  \item Develop a robust, secure and reliable operating system
    With about 340 CVS committers and thousands of contributors involved in the project
  
  \item Large documentation available, including translation for multiple languages
  
  \item Maintain developing under Revision Control System at using Apache Subversion (SVN)   
  
  \item Ensure freedom of licensing thanks to the permissive FreeBSD License  

  \item Support the project through the FreeBSD Foundation by providing financial resources, legal advice, receive donations that are destined to help developers get to conferences and events, as well as research investment and hardware purchases
	
  \item Well define the role of the people involved in the project, depending on their perspective (ISP, developer, final user, etc.)
  
\end{itemize}

%------------------------------------------------

\section{How Open Source Projects Survive Poisonous People}

%------------------------------------------------

A talk given at Google's Open Source Developer Speaker Series by Ben Collings \& Brian Fitzpatrick, who are in charge of the project Hosting Feature on Google Code Site by January 2007, in order to discuss:

\begin{itemize}
  \item A possible way to run a free open source software community
  \item Understand the community as the main resource to focus the attention
  \item Guidelines proposed by Karl Fogel in the book ``Producing Open Source Software" \cite{Fogel:2005kf}
  \item Try to reach a culture of trust in the community
\end{itemize}

\subsection{Brief description of what is meant by poisonous people at FLOSS communities}

A software development community is constantly besieged by external or even internal factors that could fracture it, even more in a FLOSS community. Factors that can be attributed to people as well as to procedures performed daily. It is extremely important to identify these elements and know both the attackers and vulnerabilities, in order to achieve sustainability of the FLOSS project.

\subsection{Main goals in Open Source Projects to survive poisonous people}

It is possible to identify four stages of protection, and specific goals for each of them are the following:

\begin{itemize}
  \item Comprehension:
  	\begin{itemize}
  		\item Understand people to protect from
  		\item Start with coding, avoid paralysis, do not be that perfectionist
  		\item Ignore noise minorities without constructive arguments
  	\end{itemize}
  
  \item Fortification:
  	\begin{itemize}
  		\item Fortify the project against poisoned people (the threat)
  		\item Keep the community resistant to infection
  		\item Maintain politeness, respect, trust and humility. The healthy of the community depends on it
  		\item Promote best practices, code-collaboration policies
  		\item Set well defined processes
  		\item Establish goals when coding, then limit the scope, avoiding limitless features loop
  		\item Rely on mailing lists, the main communication method in any FLOSS project
  		\item Do no let people open old discussions, unless it is something that really worth it
  		\item Produce documentation on mission and objectives, bug fixes, issue trackers, so new developers are able to continue with the guidelines without repeat mistakes
  		\item Define standard log formats to follow it in an easy way
  		\item Keep coordination with committers about changes
  		\item Do not allow commits of very large changes on the project's code, try to be incremental (using branches)
  		\item Identify the Bus Factor of the project: the number of developers that have to get hit by a bus to leave the project on chaos
  		\item Encourage people to spread their expertise around
  		\item Reduce territoriality
  		\item Promote equal values about the project among committers
  		\item Consider voting/veto/forks when necessary in pro of the community (as a last resource). It is healthier try to come to resolutions on relatively easy basis
  	\end{itemize}
  	
  \item Identification:
  	\begin{itemize}
  		\item Look or notice for people who really help the project to evolve
  		\item Encourage people to implement NETIQUETTE
  		\item Do not let others to blackmail the project on groundless basis
  		\item Keep the values over certain new proposals (like new unnecessary features)
  		\item Give more importance to the coding design over the code
  		\item Discuss features with the community
  	\end{itemize}
  
  \item Disinfection:
  	\begin{itemize}
  		\item Look for pushing out the people disrupting attention and focus of the project
  		\item Always asses the damage of situations and decisions
  		\item Pay attention on newcomers, letting them a chance
  		\item Look for the facts over the emotions
  	\end{itemize}
    
\end{itemize}


%------------------------------------------------

\section{SilverStripe CMS}

%------------------------------------------------

This talk was held in August 2007 by two of the founding members of the project, Sigurd Magnusson and Sam Benny. A pair of FLOSS  enthusiasts from New Zealand, who have participated in the Google Summer of Code competition. This couple claims that FLOSS is a democratic process in New Zealand and that this system is in fact reproducible.

\subsection{Brief description of SilverStripe Content Management System}

Magnusson and Benny started building websites since year 2000, and in this process they came to accumulate deep knowledge about the tasks involved in this business. They developed an application that describe as:

  	\begin{itemize}
  	  \item Bigger than a blog tool
  	  \item More nimble than existent open or closed source CMS enterprise systems
  	  \item Focus on usability for website managers
  	  \item Out of the box, and at the same time providing a framework
  	\end{itemize}

\subsection{Main goals of SilverStripe}

  	\begin{itemize}
  	  \item Use FLOSS to advertise the product and get people on board, with no ads more than the own source code
  	  \item Give a chance to FLOSS business model, as they offered the product as proprietary at first with no luck
  	  \item Rely on FLOSS as a cheaper and more effective way to market
  	  \item Use all the lessons learned and make it available to potential many other contributors
  	  \item After having based their infrastructure on FLOSS, give back something in return to communities
  	  \item Establish guidelines when managing people on a FLOSS project
  	    \begin{itemize}
  	  	  \item Establish a road-map
  	  	  \item Give public recognition
  	  	  \item Motivate the community in a proper way
  	  	  \item Delegate work when necessary
  	  	  \item Encourage group answers in reply to questions from individual users, as everybody is noticed
  	  	  \item Try to mentor herding (guidance in the process)
  		\end{itemize}
  	  
  	  \item Document mistakes and their replication, so other users/developers do not commit them again
  	  \item Participate in events such as Google Summer of Code due to Mash-ups, functionality integration from third parties is real fun and enriching
  	  \item Internationalization is an important factor to disseminate
  	  \item Provide support over tools, by services and hosting for instance
  	  \item Promote web conferences, face to face meetings are healthy and necessary
  	  \item Look for diversification of the project
  	  \item Predict what people want from the product
  	\end{itemize}

%------------------------------------------------
\section{Camino}
%------------------------------------------------

Talk held in January 2007 by Mike Pinkerton, a software engineer in MacDev at Google, who worked at Netscape from 1997 until 2002. Pinkerton tries to highlight the lessons learned in the process of develop a FLOSS project around a web browser, in a time that were only two direct competitors ``Netscape Communicator" and ``Microsoft Internet Explorer" back in 1998.

\subsection{Brief description of Camino}

It started as a Mozilla initiative to continuing developing a web browser for the Mac OS X platform, due to cross-platform was an important feature for the browser market, and it still is at this days. But in the very beginning this project was born as business strategy by Netscape, who created a business model around selling their product: a commercial web browser. The main problem faced by this company was that Microsoft started giving it for free as part of Windows OS. Anyhow Netscape realized that Microsoft's strategy could be used against them in a FLOSS field, because Internet Explorer was highly integrated with its also proprietary OS. By turning the Netscape Communicator source code available, they would be competing in the opposite direction to Microsoft's strategy, to whom opening the source code was non-viable. This is how in March 1998 was announced the release of Netscape Communicator source code as the Source 331 Project.

\subsection{Main goals of Camino as a FLOSS project}

  	\begin{itemize}
  	  \item Compete against Microsoft or defeat it in the web browsers field
  	  \item Trust in the support of the community. The lasts entire release cycles of Camino were based on the open source community, with no Netscape support at all. The quality for Camino Beta releases was very high, they didn't want to let the community down. There was a documentation group, testers, designers, bugs reviewers, etc.
  	  \item Defend values. Netscape crew had to wear two hats, one for the Corporation continuing with Netscape developing, and on the other hand with Mozilla's community
  	  \item Accept changes when required. Raptor project was another rendering engine with better performance at the time. Also it was difficult to continue the relationship with Mozilla Foundation at certain point
  	  \item Respect the community. Management decision about change the direction of the project was not properly announced, and consequently much of the credibility on the project was lost. Later when Camino project split apart, started to take Mac community more seriously
  	  \item Understand the different roles in the community. For instance there are two distinct groups, the designers (following standards), and the programmers (implementing the requirements), try to achieve a balance or ​​reduce the gap between them is crucial
  	  \item Seek evolution. Since 2000 developers started to work into improve performance, reduce memory consumption, incorporate a new rendering engine. The new develop aims to be native, fast, keep compatibility (including Cocoa API's, Raptor and Gecko)
  	  \item Keep the community motivated. Camino was one of the last effort products from Netscape for the community. Improvements such as an user friendly release and attractive website at the time of 1.0 mayor version, produced 200 downloads per day
  	  \item As Eric Raymond's Cathedral and Bazaar \cite{Raymond:1999er}: be open as the promiscuity, share as much as you can to build a really committed community, involve users and developers
  	  \item Camino did not followed the ``Release Early, Release Often" directive, with the intention of maintain high quality before releasing
  	  \item Mozilla.org was Developer-Centric (forgetting the users), but Camino followed another direction
  	  \item Camino looked for reach end-user developers, offering forums and Q\&A, having the right focus on the website
  	  \item Modules ownership is not a good practice according to Camino project, because it centers the whole development process in individuals, nevertheless it is sometimes necessary to avoid dispersion. A weak owner is worst than no owner at all
  	  \item Consider the testers as the most valuable resource. For instance, they just want a functional browser no matter what engine it has, and they will do all necessary tests to obtain it
  	  \item It is imperative to have an Open Bug Database to file code issues
  	  \item Understand that it is impossible to please everyone, it educates the community
  	\end{itemize}

%------------------------------------------------
\section{Greg Kroah Hartman on the Linux Kernel}
%------------------------------------------------

In 2008 Greg Kroah Hartman, as a subsystem kernel maintainer for USB, driver core, and previously PCI related, gave this talk about the Linux Kernel, the process, who is doing it, how it is been done.

\subsection{Brief description of the Linux Kernel ecosystem}

The 1.0 version of the Linux Kernel was released on March 1994. At this time it only supported single-processor i386-based computer systems, as well as a reduced set of hardware drivers in general. In subsequent releases, portability became a concern, so the kernel gained support for computer systems using processors based on the Alpha, SPARC, and MIPS architectures, and also began to incorporate drivers to support a large number of hardware devices. This evolution was possible because many hackers became interested in the project proposed by Linus Torvalds who licensed it as GNU GPL v2 \cite{Torvalds:1991lt}, which allowed and facilitated the rules for sharing the source code since the very beginning.

\subsection{Main goals of the Linux Kernel project}

  	\begin{itemize}
  	  \item Provide a stable-multiplatform operating system kernel
  	  \item Encourage to include code from the community into the kernel
  	  \item Keep the current development scheme without following an strict hierarchy, as it is still organized enough and efficient. Linux is not intelligent design, is evolution
  	  \item Support for new features is a day to day work
  	  \item Testing is about running the kernel and check if it still works, rely on developers and users of the community
  	  \item Encourage companies to increasingly sponsor more Linux Kernel developers
  	  \item Lead questions like if it is a good idea to let current work-founding companies impose the direction of Linux Kernel development
  	  \item ``Regression" is a way to correct mistakes
  	  \item Enterprise boxes (such as kernel stable releases) are not something Linux Kernel Community maintain, it is a matter of the companies supporting it
  	  \item Currently making emphasis on KVM, and the possibilities to control real time costs (CPU usage, execution time, network resources, etc.)
  	\end{itemize}

%------------------------------------------------
\section{Mercurial Project}
%------------------------------------------------

Continuing with the Google TechTalks series, in June 2006 Bryan O'Sullivan talks about a new project oriented to distributed revision control systems. Mercurial was started as an appropriate tool likely to FLOSS developments, allowing the evolution of such projects.

\subsection{Brief description of Mercurial}

Written in Python an C programming language, under GNU GPL v2 license, Mercurial is a cross-platform, distributed revision control tool for software developers. This project started as an alternative to Version Control Systems (VCS) such as BitKeeper, who was responsible until those days for holding the development scheme of important and robust software projects such as the Linux Kernel. Although it was efficient, communities like the one from the Linux Kernel were determined to change their proprietary VCS to a better and flexible solution, so new initiatives were born in parallel and by different people (mainly BitMover and Git) \cite{Mackall:2006mm}.

\subsection{Main goals of the Mercurial project}

  	\begin{itemize}
  	  \item Produce a reliable and sustainable Revision Control System
  	  \item Keep written in Python, be distributed, run fast with good performance, and be easy to understand, so developers can focus on their own projects
  	  \item Mercurial's major design goals include high performance and scalability, decentralized, fully distributed collaborative development, robust handling of both plain text and binary files, and advanced branching and merging capabilities, while remaining conceptually simple
  	  \item Displace competition as Subversion, Bitkeeper or Git
  	  \item Maintain a business model around cloud storage plans
  	  \item Invest efforts in efficient patch handling
  	\end{itemize}

%---------------------------------------------------------------
\section{Stefano Zacchiroli on Debian: 20 Years and counting}
%---------------------------------------------------------------

This time the current Debian Project Leader Stefano Zacchiroli by year 2013 talks about the success of the Debian project as a FLOSS community and the factors that have allowed it over the last 20 years.

\subsection{Brief description of the Debian project}

Debian itself is a FLOSS operating system formed mainly by software under GNU General Public License. It was founded by Ian Murdock in 1993, and has been continued to develop since then by a community of people who collaborate as volunteers following the Debian Free Software Guidelines (DFSG) \cite{Perens et al:2004bp}. The software, the community, and the set of rules under which they coexist, is what gives life to the project as a whole.

\subsection{Main goals of Debian as a project}

  	\begin{itemize}
  	  \item States the importance of the Debian Freedom Manifest
  	  \item Focus on developers and users
  	  \item The first idea was to turn GNU/Linux competitive to existent commercial OS's
  	  \item Provide an easy method to install
  	  \item Build software in a collaboratively way by experts of many different areas
  	  \item Be the first mayor distro developed openly in the spirit of GNU
  	  \item Count with the support of the Free Software Foundation
  	  \item Innovate at introducing GNU/Linux distributions in a collaboratively way
  	  \item An OS meant to last on time, installable on Servers and Desktops
  	  \item 1/3 Debian: the OS
	  	\begin{itemize}
	  	  \item Produce stable product versions looking for binary distributions, completely free, released every 24 months, supporting many architectures, provide repository archiving and security support (3 years)
	  	\end{itemize}  	  
  	  \item 1/3 Debian: The Project
	  	\begin{itemize}
	  	  \item Common Main Goal: Create the best possible Free OS. Based on the following documents
	  	  \item Debian Social Contract (1997): agreement between no only developers, but among the whole ecosystem\\
	  	  		. Releases 100\% FLOSS\\
	  	  		. Give back contributions to community\\
	  	  		. Do not hide problems\\
	  	  		. Establish priorities: users and FLOSS
	  	  	\item Debian Constitution (1998):\\
	  	  		. Define the rules of a Free Software-compatible democracy\\
	  	  		. Being a strong motivation for people who join to the project
	  	\end{itemize} 
  	  \item 1/3 Debian: The Community
	  	\begin{itemize}
	  	  \item Transparency as a principle
	  	  \item Provide empowerment to members of the community. Based on the public nature of the source code and the impact that any participant is able to produce with the proper skills
	  	  \item Maintain high volumes of communications: mailing lists, IRC, web services (social network principles-compatible like identi.ca)
	  	\end{itemize}
  	  \item As the project evolved, the goals aimed at:
	  	\begin{itemize}
	  	  \item Produce a faster and robust dependency-based boot system
	  	  \item Provide a completely free Linux Kernel, excluding proprietary firmware
	  	  \item Offer a large choice of pure blends or customized distributions (education, medicine, science, etc.)
	  	  \item Rely on online services supporting user demands
	  	\end{itemize}
  	  \item Nowadays:
	  	\begin{itemize}
	  	  \item Maintain a multi-architecture (cross-compilation, package share among architectures, mixed 32/64 bits binaries)
	  	  \item Include Private Cloud Support (OpenStack, XEN) and Public Cloud Support (EC2, Azure)
	  	  \item Handle more end-user desktops (Gnome, KDE, Plasma, XFCE)
	  	  \item Guarantee easy system upgrades
	  	  \item Support new architectures: armhf (such as handheld devices), s390x (virtualization servers)
	  	\end{itemize}
  	  \item Debian Package Quality:
	  	\begin{itemize}
	  	  \item Keep the culture of technical excellence
	  	  \item Continue the package design, testing and maintenance scheme
	  	  \item Release Mantra: ``We release when is ready"
	  	\end{itemize}  	  
  	  \item Attachment to freedom
  	  \item Independence (not commercial companies behind)
  	  \item Defined decision making rules:
	  	\begin{itemize}
	  	  \item Do-ocracy, where individual developers are able to make any decision regarding their own work
	  	  \item Democracy through a voting system
	  	  \item Consequences:\\
	  	  		. Reputation follows work\\
	  	  		. No benevolent dictator, button-up scheme\\
	  	  		. No imposed decisions from power figures (such as corporations)	  	  
	  	\end{itemize} 
  	  \item Recently there are many derivative distributions that depend on Debian, and this is a way to reach new potential users/contributors (everybody wins)
  	  \item The intention is to guarantee the sustainability of the Free Software Ecosystem
	  	\begin{itemize}
	  	  \item Return in back (reduce patch flow viscosity)
	  	  \item Give credit when apply
	  	\end{itemize} 
  	  \item To contribute, there are standards to accomplish
	  	\begin{itemize}
	  	  \item It is not about provide a package and leave
	  	\end{itemize}
  	  \item Receive punctual contributions by monetary donations, intended for resources, face to face meetings, conferences, etc.
  	  \item Promote work-contribution:
	  	\begin{itemize}
	  	  \item End-users: testing, reporting, fixing bugs, monitoring packages
	  	  \item Developers: packaging new contributions
	  	  \item Non-development tasks: translation, themes design, media diffusion, documentation, events, etc.
	  	\end{itemize}
	  \item Make it clear that join Debian imply a commitment
	  \item A recent Diversity Statement was recently published to help increase the quality of the community, adopting all kind of contribution and contributors meeting the Debian Freedom Manifest  	  
  	\end{itemize}


%------------------------------------------------
\section{The EdgeBSD Project}
%------------------------------------------------

Pierre Pronchery gives this speech as EdgeBSD project leader at the FOSDEM 2014 event in Belgium. An interesting initiative with the aim of expanding the potential of the NetBSD system, but in a parallel or separate way, due to the difficulty to contribute to the latter.

\subsection{Brief description of EdgeBSD project}

Pronchery claims that his intention is not to fork NetBSD, but he also wants to add many features to this system in the form of contributions, but has not been able to do it because of the limited conditions imposed by current maintainers of the NetBSD project. This situation leads him to produce his own version of system (EdgeBSD), which is based on NetBSD, and intends to remain this way even in upcoming releases in relation to the system core.

\subsection{Main goals of EdgeBSD as a continued FLOSS project}

\begin{itemize}
	\item As a Fork of NetBSD, EdgeBSD claims that its predecessor harms development, due to its cathedral-like type of development
	\item The Version Control System in NetBSD (CVS) produces a manual and tedious work, so EdgeBSD looks for improve that
	\item Reduce the many restrictions that are present in NetBSD, even to official developers
	\item Maintain source packages from NetBSD in specific branches, letting the users to pull from them if desired
	\item Some changes introduced in comparison with NetBSD are:
		\begin{itemize}
			\item Decentralized VCS with Git
			\item Less restrictive in order to push contributions
			\item More people involved due to its flexibility
		\end{itemize}	
	\item Rely on a decentralized tool to maintain VCS, without falling into discussions over it, and giving importance to the project mainly
	\item Propose an improved development model
	\item Test and enable more features in the system
	\item Encourage the community for testing of the core code and packages
	\item Provide a more useful way to share patches and fixes in general
	\item Provide a fun and attractive as a research and development platform while delivering a modern, robust, and industrial-grade system for all ranges of computer devices
	\item Provide more collaboration services (Email, Calendars, Secure IM, VoIP, Archive Hosting, etc.) for developers in order to attract them. Now including a graphical installer, ready-to-flash images for range of devices, continues updates
	\item Provide support for hand-held devices
	\item Push for more infrastructure and automated procedures, in order to optimize the development process
	\item Try to industrialize the FLOSS development process, maybe with support of private companies, as well as to attract companies vendors to use the system
	\item Try to provide a proper release engineering, already based on stable release of NetBSD 6
	\item Pushing up the Add-work to promote the project
	\item Provide also binaries for some architectures, although EdgeBSD is compilable on any
	\item Make new package versions regularly, associated to maintainers from the community
	\item Unify developer and contributor efforts in a centralized community, archives or mirrors, so contributions are not lost anymore as it currently happen in NetBSD
\end{itemize}


%------------------------------------------------
\section{Secrets what I wish I knew before joining Debian}
%------------------------------------------------

This talk was given at the Free and Open Source Development European Meeting (FOSDEM) in 2013 by Lucas Nussbaum, the current Debian Project Leader. This is an European event centered around FLOSS development, aim to enable developers to meet and to promote the awareness and use this kind of software and form or style of development. Nussbaum discuss about some well defined patterns in the Debian community in its current status.

\subsection{Brief description of some patterns identified in Debian project}

As a 20 year old project, people from Debian's community have identified some typical and recurring behaviors that do not help the overall strengthening of the project. Figures such as the project leader are aware of this phenomenon, and make it public with the purpose that the members of the community could be able to avoid them. Factors ranging from the rules governing the project, people involved and their motivation, to development models that are followed, revealing the inner workings of Debian.

\subsection{Main goals of Debian community to identify patterns}

  	\begin{itemize}
  	  \item The establishment of a do-ocracy ruled by a technical committee
	  	\begin{itemize}
	  	  \item In general maintaining a culture of technical excellence
	  	  \item Should not overrules the maintainer of a package
	  	  \item Is a long process, could take months to make a decision
	  	  \item Rarely goes to voting, considered as a last resource
	  	  \item Core teams take care of key areas of the project, taking good decisions most of the time
	  	  \item When not, it results in flamewars, hurting the project badly
	  	\end{itemize}
  	  \item The presence of Benevolent Dictators
	  	\begin{itemize}
	  	  \item Eventually there exists some territoriality in certain Debian areas
	  	\end{itemize}
  	  \item File a bug and send a patch philosophy 
	\begin{itemize}
	  	  \item In general for small bugs, it is not efficient
	  	\end{itemize}
  	  \item Nobody feels empowered to do the work
	  	\begin{itemize}
	  	  \item No strict hierarchy, possible blocking decisions by not knowing the leadership
	  	  \item Encourage to do it, anyway it can be reversed
	  	\end{itemize}  	  
  	  \item Nobody wants to do the grunt work
	  	\begin{itemize}
	  	  \item Work in terms of other areas but computers. It is recently changing (communication, accounting, marketing, etc.)
	  	\end{itemize}
  	  \item Over optimized processes
	  	\begin{itemize}
	  	  \item Some developers do it for themselves, but for many it is possible to rely on Bug Tracking Systems
	  	\end{itemize}
  	  \item Button-up designs and adoption
	  	\begin{itemize}
	  	  \item Debian follows a decentralized development model
	  	  \item This process generally starts by trying to solve own problems, then expands to other teams, finally to the whole project. It is a long process, eventually it takes months or years
	  	  \item Package Tracking Systems (PTS), a Centralized Source Package Dashboard, turns increasingly imperative due to the diversification of the project
	  	  \item Developer Packages Overview (DDPO), a Centralized Maintainer Info, due to overwhelming number of contributions
	  	\end{itemize}  	  
  	  \item It is necessary to keep looking at changes over time
	  	\begin{itemize}
	  	  \item snapshot.debian.org provides a list archive for packages in previous Debian releases
	  	\end{itemize}
  	\end{itemize}

\pagebreak

%----------------------------------------------------------------------------------------
%	BIBLIOGRAPHY
%----------------------------------------------------------------------------------------

\begin{thebibliography}{99}


\bibitem[Fogel, 2002]{Fogel:2005kf}
Fogel, Karl.
\newblock ``Producing Open Source Software: How to Run a Successful Free Software Project".
\newblock O'Reilly Media, 2005. p. 1-202.

\bibitem[Mackall M., 2006]{Mackall:2006mm}
Mackall, M. (2006).
\newblock ``Towards a Better SCM: Revlog and Mercurial".
\newblock Linux Symposium Proceedings. Ottawa: Selenic.

\bibitem[Perens et al., 2004]{Perens et al:2004bp}
Perens, Bruce, et al.
\newblock ``Debian Free Software Guidelines".
\newblock Debian community. Version 1.1, 2004.

\bibitem[Raymond E., 1999]{Raymond:1999er}
Raymond, Eric S. (1999).
\newblock The Cathedral \& the Bazaar.
\newblock O'Reilly, pp.279.

\bibitem[Torvalds L., 1991]{Torvalds:1991lt}
Torvalds, Linus. (1991).
\newblock Linux Kernel GNU GPL License v2:  ``Linux Kernel Copying".\\
\newblock DOI=http://git.kernel.org/cgit/linux/kernel/git/torvalds/linux.git/tree/COPYING
 
\end{thebibliography}

%------------------------------------------------

\end{document}

%------------------------------------------------
% Format Templates:
%\begin{figure}[H]
%\center{\includegraphics[width=0.5\linewidth]{placeholder}}
%\caption{Example image.}
%\label{fig:speciation}
%\end{figure}
%------------------------------------------------
%\begin{wrapfigure}{l}{0.4\textwidth}
%  \begin{center}
%    \includegraphics[width=0.38\textwidth]{fish}
%  \end{center}
%  \caption{Fish}
%\end{wrapfigure}
%------------------------------------------------
%\begin{description} % Numbered list example
%\item[First] \hfill \\
%\lipsum[9] % Dummy text
%\end{description} 
%------------------------------------------------
