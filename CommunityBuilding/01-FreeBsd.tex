\documentclass[11pt]{article}

\title{\textbf{CASE STUDIES II: FreeBSD Project Community}}
\author{Sergio Arroutbi Braojos}
\date{\today}
\usepackage{listings}
\addtolength{\voffset}{-50pt}
\begin{document}

\maketitle

\section{Introduction}
On this Tech Talk, Robert N.M. Watson, from Cambridge University, talks about FreeBSD project, and how the community behind this project works. Questions as what is FreeBSD and FreeBSD project, what do you get when you get a FreeBSD distribution, or how does the FreeBSD project work in terms of contributors or licensing are responded on this talk, which is not focused in any particular technical feature of the project, but rather on its general way of work.
\section{FreeBSD Foundation}
Behind FreeBSD project, the FreeBSD Foundation exist, in order to support the project on different ways, but always without being implied on the technical decissions happening on the project, but rather with:
\begin{itemize}\itemsep0pt
\item{Sponsor development}
\item{Hardware purchase}
\item{Collaborative R\&D agreements}
\item{Developer travel grants and sponsorship}
\item{\textbf{Sponsor the project economically, by obtaining donations}}
\end{itemize}
\section{Project Necessities}
On the talk, apart from exposing all the stuff that the project ``produces'', such as the FreeBSD kernel, the releases, the ports collection (packages), the documentation or support. However, on this talk, it is alsoexposed what the project ``consumes'', as a way of identifying the project necessities, to be pointed out the following:
\begin{itemize}\itemsep0pt
\item{\textbf{Hardware}}. Donated and sponsored, specially racked already prepaired hardware.
\item{\textbf{Bandwidth}}. Needed in vast quantities, as in any other project, but even more for Operating System Open Source projects, where, every six months in this case, lots of people from around the world download tons of terabytes to test the new releases.
\item{\textbf{Financial Resources}}. Travel grants, salaries, contracts and other type of grants.
\item{\textbf{Gratitude}}. Thanks, good press, user testimonials or success stories are welcomed as well.
\end{itemize}
\section{Community on FreeBSD Project}
\subsection{Committers} 
People with commit rights. They come from different 34 countries, from 6 distinct continents. The main development groups are in Nort America, Europe and Japan. There is also an important group of contributors from India and Australia.\\
\\
Regarding ages, 30 to 40 years old group is the most large, with a mean age of 32.5 and a median age of 31.\\
\\
In terms of occupation, there is a group of heterogeneous jobs such as professional programmers, hobbyists, consultants, university professors, researches or students.\\
\\
Committers must have certain key characteristics:
\begin{itemize}\itemsep0pt
\item{Technical expertise}.
\item{History of contribution}.
\item{Ability to work well in the project community}.
\end{itemize}
\subsection{Ports Committers and Maintainers} 
They are responsible of maintaining the binaries of the different software pieces available in the project.
In 2006, 158 port committers existed, and over 1400 port maintainers for 16600 ports were active in the project. As average number, 85 ports/committer, 9 ports/maintainers and 8 maintainers/committer existed.
\subsection{Mentors} 
As in other open source projects, mentors support people starting in the community.
They propose new people to core group, or other group. A key factor is that people, apart from demonstrating the technical abilities, demonstrate how commited are with the community.
\section{Events}
There are two type of events regarding FreeBSD Project:
\begin{enumerate}\itemsep0pt
\item{\textbf{Conferences}}. BSDCon, MeetBSD, EuroBSDCon, AsiaBSDCon, BSDCanada, etc.
\item{\textbf{Developer Summits}}. Two Day Events, normally three a year, one in North America, one in Europe, one in Asia. e.g.: in 2007, conferences in Ottawa, Tokyo and Copenhagen occurred. Focused in development and developers to know each other.
\end{enumerate}
\section{Conflict Resolution}.
Developers are normally common sense, easy to cooperate, independent and neutral. The conflict resolution is normally avoided by intentionally skipping overlap between members of the community.\\
\\
However, strong technical disagreements will occur, involving technical and personal conflicts. If conflicts get out of hand, normally a member from the core group mediates to avoid flames.\\
\\
There is a special kind of conflict, the ``Bike Shed'', that appeared for the first time on this community, and that is a good example on a type of conflict that must be avoided, as this kind of conflict consist normally around unimportant stuff that is handled with many strong opinions from many different people in the community, meaning lots of discussion as well.
\section{Conclusion}
Free BSD is one of the largest, oldest, and most successful Open Source projects, built on millions of lines of source code from hundreds of committers and thousands of contributors. There are tens of millions of deployed systems hosting FreeBSD Operating System. The highly successful community model makes this community a very desirable project for any Open Source advocate to participate. Join the community!!!
\end{document}
