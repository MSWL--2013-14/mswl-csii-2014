\documentclass[11pt]{article}

\title{\textbf{CASE STUDIES II: Debian Secret what I wish I knew before joining Debian}}
\author{Sergio Arroutbi Braojos}
\date{\today}
\usepackage{listings}
\usepackage{cite}
\usepackage{booktabs}
\addtolength{\voffset}{-50pt}
\begin{document}

\maketitle

\section{Introduction}
On Tech Talk, dated on February 2013 as part of FOSDEM 2013 ~\cite{FOSS00}, Lucas Nussbaum, who is the current Debian Project Leader (DPL), gives a presentation on those aspects that we would like to have discovered before joining Debian project.\\
\\
From the comunity perspective, this lecture faces very interesting issues, e.g.:
\begin{itemize}
\item{Decission Making in Debian Project}.
\item{\textbf{Debian Patterns}}.
\end{itemize}
Debian determines that decision making is based in two fundamental aspects:
\begin{enumerate}\itemsep0pt
\item{Do-ocracy}. In this project, as in 99\% of the Open Source Projects driven by comunities, \textbf{reputation follows work}, and never the other way round.
\item{Democracy}. There is no benevolent dictator, as in other Open Source Projects. Decission making is not imposed.
\end{enumerate}

But how does this work in practice? What makes Debian a so unique, ``but sometimes frustrating'', project? Debian follows some patterns in order to accomplish the project decission making strategy. In this talk, Debian DPL provides some key aspects in order to clarify why, despite being a project considered to be one of the most democratic ones, ``Debian is not a rant'', and not everybody is continuosly flooding with their opinions on how to drive the project, make the decissions or follow a particular strategy.

\section{Debian is magic}
DPL remarks that the \textbf{main value} existing on Debian has to do with its \textbf{Task Force}. Debian Project is marked because of its great community, considering that:
\begin{itemize}\itemsep0pt
\item{Great experts on computing science, working on a volunteer basis}.
\item{Project is under continuous improvement, and making these improvements free for everybody}.
\item{Culture of technical excellence.} Debian is driven by a ``Quality First'' strategy, where lots of thing can be learnt just by following mail lists.
\item{No boss.} A figure of a DPL exists, but just to give a vision, meaning no obligation to follow it. There is also a Technical Committee, but it is only a last resort option.
\item{Special Organization.} Debian is based on a majority of people who are not paid by any company. This gives a strong power in terms of independency and community driven decission making.
\item{Lots of subgroups and subprojects.} ``This involves an interesting mix of problems and challenges'' from lecturer's perspectives.
\end{itemize}
The technical talk is mostly focused in last two points, in order to clarify how Debian Project is organized, and the different issues that the community faces along the time.

\section{Do-ocracy}
What is do-ocracy. It is a very simple concept for the lecturer: \textbf{``Those who do, decide''}. However, those who do, decide on both their work, but also about the other's work. This aspect can drive to several issues:
\begin{itemize}\itemsep0pt
\item{Usual problem}: A wants B to do something. B disagrees on doing so.
\item{Solution}: \textbf{Technical Committee} (TC). This committee is the only one who can overrule this kind of situations. The decission takes long, usually months, rather than weeks. Although, normally, TC does not overrule the maintainer (B in this case). TC is compounded by eight people, normally proposed by the DPL. They are not normally needed. 2010 and 2011 needed no TC action. On 2012 just a vote about multiarch packaging tool, dpkg, was needed.
\end{itemize}

\section{Patterns}
\subsection{Benevolent Dictators, but still dictators}
There is a typical scenario where conflict appears. It has to do with maintainers who mostly work correctly. He usually works alone, without belonging to a team. Normally, he/she insists on ack'ing all decissions. There is normally a source of disagreement with other Debian Developers (DDs). They are normally related to:
\begin{enumerate}\itemsep0pt
\item{Design Decissions}.
\item{Things not done on time, quickly enough}.
\end{enumerate}
However, this kind of situation has to do with \textbf{people complaining, but normally not offering help}. Normally, their proposes is even worst.\\
\\
In this kind of situation, the maintainer must drive the situation by making everything public, using  ``BTS'' or ``civil mail list'', and being patient. Apart from that, the conflict must be handled with \textbf{focus on technical issues, offering solutions such as patches or any other type of help, and avoiding personal offense}.\\
\\
There are also some procedures to follow in other situations. For example, when opening a Bug related to packaging, and starting a conversation around the bug, it is important to CC about that particular bug for people to understand, quickly, what the mail is referred to.\\
\\
Conflicts are not that common. Normally, package maintainers take good decissions, and none of previous situations must be faced.

\subsection{Going Further: Core teams}
Debian project has a set of ``core teams''. This kind of teams take care of project key areas, such as archive (ftpmasters), testing (release teams), machines (DSA), security, and so on and so forth.
These teams are characterized by some aspects:
\begin{itemize}\itemsep0pt
\item{The take very good decissions most of the times}.
\item{However, when conflicts appear, it results on \textbf{big flamewars}}. The recent example is squeeze release. Release date was announced months after announcement, what resulted on hurting the project.
\item{Core teams are powerful}. This result on ``partitions of powers'' inside the project. They are not easily convincible when critical changes are decided.
\end{itemize}

\subsection{Please, send a patch}
As a do-ocracy driven community, there is a typical situation appearing sometimes:
\begin{itemize}
\item{A wants B to do something}.
\item{B responds: ``Please, send me a patch''}.
\end{itemize}
This demand is normally very reasonable, but other times it is not so. A very basic situation, for example, is when B made a typo in a man. There is no point for B not to fix that. In the end, \textbf{optimizing the taskforce must prevail}. If B has already performed the clone, just needs to fix and submit. Maybe, A, has to clone, be providen with access, etc. 

\subsection{Nobody feels empowered to do the work}
Normally, a common issue is that people working on a certain team do not know their role inside the team, as most teams don't have a strict hierarchy.
\begin{verbatim}
"Normally, people are Team Leaders, but do not know it. This means normally to be blocking decissions, by just not responding to an e-mail"
\end{verbatim}
There are two solutions in order to avoid this kind of situation:
\begin{enumerate}
\item{Give always feedback to those who ask for it.}. The sooner, the better.
\item{Just do it.} Normally, it can be reversed anyway.
\end{enumerate}

\subsection{Nobody wants to do the grunt work}
There is some kind of work that, despite of being grunt, must also be done. However, Debian community member is normally a computer geek, that does not want to do this other kind of projects. Debian is willing to receive other kind of geeks, such as publication communication geeks, accounting geeks, documentation geeks, etc. However, Debian is making progress in this area. 

\subsection{Over-optimized process}
DD's are normally selfish. They normally design the most appropriate procedures for themselves, which are not necessarily the best for the users or the rest of the community.\\
\\
Bug Tracking System (BTS) is dedicated to track the process and decide on actions such as orphaned packages, packages that must be removed, sponshorsip requests, and so on, and so forth. However, patches have being stopped to be tracked with BTS.

\subsection{Button-up design and adoption}
Debian Project follows the characteristic of ``Bottom-up'' design, and in the end, decission making. This is normally common to all the Open Source projects, and in Debian, due to the independency that rules the community, even more. In the end, the project follows some common characteristics:
\begin{itemize}\itemsep0pt
\item{It is decentralized.}
\item{Most innovations and contributions start on a ``scratch-an-itch mode''.}
\item{Process is long.} Sometimes, innovations last months, even years, to expand to other areas and the rest of the project as well. The typical example is Debian services, 
\end{itemize}
From lecturer's perspective, this bottom-up way of working, focused on decentralized groups with no hierarchy and strong autonomy, favours the innovation.

\subsection{Packaging practices inside teams}
There are a lot of teams who are involved in packaging, but in the end they have not many different needs. Their needs are normally related to optimizing packaging, migrating from SVN to Git to favour decentralized development, manage patches, etc.\\
\\
However, it seems that, due to the fact that the teams are decentralized and autonomous, there are as many workflows as teams. Apart from that, workflows are not normally documented. This does not ease the fact of people moving between teams, as working in different teams mean learning different workflows.\\
\\
There is a need inside the community to fix this. \textbf{Development workflows must be designed in order to decrease entry barriers and favour documentation}.

\section{Conclussions}
After presenting Debian patterns, DPL provides some information with graphics having to do with the 
Last, but not least, conclussions of the talk are provided by the lecturer:
\begin{itemize}\itemsep0pt
\item{\textbf{Several patterns.}} Already been described in previous sectio, those patterns must be taken into consideration to understand Debian way of working.
\item{\textbf{Debian particularities.}} Being Debian so decentralized and independent makes the project to be interesting and have patterns not possible on other more vertical organizations.
\item{\textbf{Cultural excellence, technology geeks, restitution.}} Why not joining?
\end{itemize}

\bibliographystyle{unsrt}
\bibliography{09-DebianIWishIKnew}{}
\end{document}
