\documentclass[11pt]{article}

\title{\textbf{CASE STUDIES II: 20 Years of Debian}}
\author{Sergio Arroutbi Braojos}
\date{\today}
\usepackage{listings}
\usepackage{cite}
\usepackage{booktabs}
\addtolength{\voffset}{-50pt}
\begin{document}

\maketitle

\section{Introduction}
On Tech Talk, dated on March 2013 as part of the NYLUG (New York Linux User Group), Stefano Zacheroli, Debian Project Leader of Debian, talks about this exciting project, how the community of Debian has evolved over its 20 years of existance, the main aspects of the project and the future expectatives. 

\section{Free Software}
To start with, Zacheroli raises a question to the Audience: ``Why Free Software?''. And his response to the question is simple, as he considers it is a matter of Freedom from the perspective of having the tools being used under control:
\begin{verbatim}
"It is a matter to have the control. The most control
you have over the software running on your electronic
devices, the more confortable you feel"
\end{verbatim}

\section{Debian}
Debian is, by 2013, up to 17000 package sources with releases of 3000 new versions of software per month. This means a challenge in terms of distribution. Debian tries to be the glue between people who make the software, and people who use the software. Obviously, what Debian proposes, is to \textbf{enhance the ease of use of Free Software in order to grow the community of users of this kind of software}. That is the main mission of Debian, and it is the main reason of Debian to be considered the most important distribution from the community building around Open Source Software perspective.\\
\\
By 1993, Ian Murdock announced the release of Debian as a new GNU/Linux distribution. The main characteristics of the distribution are:
\begin{itemize}\itemsep0pt
\item{Open Source OS (Operating System) competitive with comemercial OS}
\item{Easy to Install}. No doubt, the key aspect for Debian to succeed. This is a key factor for community to start, 
\item{Buit collaboratively}. No doubt, the key aspect for Debian to succeed. This is a key factor for community to start, 
\item{FSF supported distro for a while}.
\end{itemize}

\section{Debian stable - Squeeze}
Main product of Debian, it is the stable version of its operating system. From the community perspective, the latest stable version, which was Squeeze by that time, had certain particularities:
\begin{itemize}\itemsep0pt
\item{Binary Distribution}. The distribution is provided as a set of binaries, in order to make quicker the installation process.
\item{Release every 24 months}. Which is a particularity, as it escapes from the ``release often, release early'' aspect of Open Source. The main aspect for doing this is that Debian is ``Quality Focused'', and has been considered the most stable GNU/Linux distribution along the time.
\item{More than a dozen architectures}. The most architecture it is supported means also the most target users as well.
\item{Pure Blends}. Pure Blends are a subset of Debian that is configured to support a particular target group out-of-the-box. Medicine, Chemistry, GIS or Education are examples of Pure Blends.
\item{Popularity Statistics}. It is also remarkable how spread is the use of Debian. 1 out of 10 Web Servers use it. Apart from that, it is considered the most popular GNU/Linux distribution.
\end{itemize}

\section{Next Debian stable - Debian Wheezy}
Zacheroli gives an explanation of what was, by March 2013, the next Debian stable to be released, called ``Wheezy''. The most important factors, in terms of community building, were next ones:
\begin{itemize}\itemsep0pt
\item{MultiArch Focus}. Focus on 3rd party easy cross-compilation.
\item{New Archs}. E.g: armhf, s390x.
\item{Desktop}. GNOME, KDE, XFCE recent versions support.
Apart from that, what is more remarkable is that Debian has started to include built-in Cloud Computing support, now that this incipient technology is so popular:
\item{Private Cloud Support}. Openstack and Xen/XCP packages.
\item{Public Cloud Support}. EC2 and Azure packages.
\end{itemize}

\section{Debian Project}
The most important aspects covered by the lecturer on this Tech Talk regarding the community lays around the objectives of the project, as well as the guideline documents to follow from the community:
\begin{itemize}
\item{\textbf{Mission}}. The mission of the project is quite clear: \textbf{Create the Best Free Operating System}.
\item{\textbf{Debian Social Contract}}. Basically consists of the main guideline to be followed by all the members of the projects and the Open Source Ecosystems as a whole. It consist basically on:
\begin{enumerate}\itemsep0pt
\item{100 \% Free Software}. As the rule of thumb.
\item{Don't hide problems}. As a way of Commitment.
\item{Give Back}. As a way of Restitution.
\item{Priorities: Users and Free Software}. As a way of living.
\end{enumerate}
\item{\textbf{Debian Constitution}}. It was written on the project constitution. It gives the project a ``Nation Nature'', with its own constitution, which defines the Structures and Rules of a Free-Software compatible democracy.
\item{\textbf{Community Members}}. Lecturer clarifies how strong Debian Project community is. By 2013, more than 1000 project members world-wide are part of this exciting project. Developers are based basically in Europe and North America (as 99\% of Open Source Projects), with increasing contributor number from South America. 
\end{itemize}

\section{Debian Community}
Following aspect faced by lecturer in the talk is around Community of Debian. From his perspective, Debian Community is characterized by three main aspects:
\begin{itemize}
\item{\textbf{Open Development}}. Basing development and packaging on Open Source means easier way to discover and fix the problems. ``Do not hide the problems'', as a rule of thumb of the daily work, and ``Show me the code'' as the best way of showing transparency and look for solutions to problems arising.
\item{\textbf{Communication}}. Mailing lists, IRC, social networks (Debian on identi.ca), Web Services and other tools are the channels for community members to cooperate with each other.
\item{\textbf{Large number of Tech-savvy users}}. 
\end{itemize}

\section{Debian Distribution Particularities}
Later, Zacheroli talks about a very important aspect, which is how Debian, nowadays, survives the so competitive market that GNU/Linux distributions mean. In 1993, no more options were available, but in 2013, twenty years later, more than 300 Distros, according to ~\cite{DISTW00}.\\
\\
In this aspect, why is Debian important for the comunity and what is the reason for its popularity? The lecturer emphasizes on those aspects that make Debian to be unique compared to other distributions:
\begin{itemize}
\item{\textbf{Package Quality}}. Debian is based on the ``Culture of Technical Excellence''. Package maintainers are software experts, in the end. The project leader asserts:
\begin{verbatim}
"We release when it is ready"
\end{verbatim}
\item{\textbf{Attachment to Freedom}}. Debian is committed to Software Freedom, not only about the software being distributed, but also in other aspects such as:
\begin{itemize}\itemsep0pt
\item{Firmware}. Software released is free in 100\%, and Firmware is too.
\item{Infrastructure}. Debian uses no non-free web services, as well as no non-free tools or infrastructure.
\end{itemize}
Previous aspects drive on a \textbf{Community Awareness}, where users are conscious of Debian not to betray Free Software principles, and meaning a \textbf{high bar for Free Software advocates}.
\item{\textbf{Indepence}}. Debian is proud to be a company non-dependant project. This is remaarkable nowadays, where companies are so involved in Open Source Project environment. Debian survives on donations and gift-economy. This helps on Debian credibility around decission making, as Debian decissions rely on a ``non-profit'' strategy.
\item{\textbf{Decission Making}}. Decission making on the project is based on two main aspects:
\begin{enumerate}\itemsep0pt
\item{\textbf{Do-ocracy}}
\item{\textbf{Democracy}}
\end{enumerate}
Both aspects mean that the comunity reputation is based on the work performed, there are no benevolent dictators, and above all, \textbf{Decissions are not imposed}.
\end{itemize}
It is important to consider this set of aspects in order to understand how this project has acquired a so well considered reputation, and how Open Source Project should chose a sameless politics if they want to reach to have a commited and happy community.

\section{Debian Derivatives}
Zacheroli faces the issue of other Distribution which are Debian based. The most popular one is Ubuntu, which in some aspects is much more popular than Debian. However, it is remarkable how the lecturer considers derivative works a normal thing on Open Source project, as it is considered in the fourth freedon around Free Software, related to derivative works.\\
\\
Debian Project leader emphasizes how rather than being a danger for the project, Ubuntu has supposed sinergies for the Debian Project as well, as many patches from this distribution have been contributed back as well.

\section{Contribution to Debian}
Finally, Zacheroly takes an opportunity to aim the audience to contribute to the project. The most remarkable aspect here is how Project Leaders must also perform the role of Comunity Managers, helping on non technical aspects such as donation asking can be. Contributions to Debian can be done in different ways:
\begin{itemize}\itemsep0pt
\item{Hardware Donations}. 
\item{Hardware-Related Services}. Hosting, etc.
\item{Money}. Above all, for sponsor development meetings.
\end{itemize}
Apart from that, there are other ways to contribute, having not to do with the monetary thing, but rather with aspect such as:
\begin{itemize}\itemsep0pt
\item{Downloading, Using, Testing Debian, as well as reporting Debian bugs}. 
\item{Adopt Debian orphaned packages}.
\item{Join packages team}. There are different teams, organized around programming languages, fields of use, etc. 
\item{Hack on Debian infrastructure}.
\item{Work on non-development stuff}.  Works as translations, design (websites, themes), communication, marketing, documentation, accounting or events organising are key aspects for project to succeed.
\end{itemize}
As the last remarkable aspect about joining Debian community, Zacheroli encourages the audience to reflect on the kind of commitment and knowledge that they can provide, and acquire the role appropriated according to this.

\bibliographystyle{unsrt}
\bibliography{07-20YearsOfDebian}{}
\end{document}
