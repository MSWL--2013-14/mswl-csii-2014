\documentclass[11pt]{article}

\title{\textbf{CASE STUDIES II: Silver CMS, from New Zealand}}
\author{Sergio Arroutbi Braojos}
\date{\today}
\usepackage{listings}
\usepackage{cite}
\usepackage{booktabs}
\addtolength{\voffset}{-50pt}
\begin{document}

\maketitle

\section{Introduction}
This speech is provided by Zigurd Magnusson, mainly, and Sam Minnée, secondarily, both belonging to SilverStripe company, on 1 August 2013, for Google Tech Talks. The talk is centered, on the one hand, in software production on New Zealand, where is placed SilverStripe company. From the community perspective, knowing different zones of the world is important, to characterize the social and economical environment where FLOSS is produced.\\
\\
On the other hand, the talk is centered on SilverCMS, the reasons why SilverStripe decided to launch this CMS as an Open Source project, and how the mission of managing people becomes a particular aspect on this kind of project.

\section{New Zealand}
Zigurd Magnusson starts by introducing the country where both lecturers come from, New Zealand. Regarding the country, the speaker highlights some characteristics from the country, for instance:
\begin{itemize}\itemsep0pt
\item{Population}: 4 million people (much less than the important states of USA). Similar to Oregon population.
\item{History}:Lecturer describes New Zealand as an "Old country". Settled 800AD with pacific islanders. European discovery from 1634, received massive immigration from 1840.
\item{Considered to be a liberal country with liberal politics}. The Prime minister (August 2007) was a woman.
\item{It has a small and remote economy}: The economy is, in fact, very dependant on exports. For that reason, it is a country which must innovate in fields like:
    - Wine, Tourism or Movies...
    - ... and Of course, Software.
\end{itemize}
Regarding the main New Zealand Software production, speaker talks about next contributions:
\begin{itemize}\itemsep0pt
\item{Jade}: A programming language that provides a user interface to create classes and carry out other programming stuff.
\item{Aftermail}: A company that provided email management platform.
\item{Xero}: A company that develops SaaS (Software as a Service).
\item{Peter Guttman}: Who has worked with PGP.
\item{PlanHQ}: A web based Business Plan Software.
\item{Catalyst IT}: A company that provides services around Open Source Software, such as Moodle, Drupal, Mahara or Kora.
\item{The most important use of FLOSS in New Zealand is the voting system}: This system is an application written in Perl/Mason on top of a PostgreSQL database.
\end{itemize}

\section{Silver Stripe}
Silver Stripe is the company that the speakers belong to. The company was, in 2007, composed of 18 people, and is, basically, dedicated to get money by providing web building services.
As a rule of thumb, the company staff had to dedicate 20\% of their time to the Open Source products of the company.
The main product of the company is SilverCMS, which is a Content Management System Open Source Project that provides a product that:
\begin{itemize}\itemsep0pt
\item{Is a more complete tool than a blogging system}.
\item{Focused on usability for the site manager}.
\item{Out of the box Framework, which provides a complete solution but with an easy installation process}.
\end{itemize}
In 2011, the system was the 15th most downloaded Open Source CMS, as shown by the Open Source CMS Market Share report ~\cite{CMS00}
Besides this, the speaker describes why SilverStripe took the decission of developing their main project as an Open Source Project. The main reasons to do so where:
\begin{enumerate}\itemsep0pt
\item{Transparency}: advertising the product by showing it (promote downloads, demos, etc.) rather than advertising it.
\item{Way to market}: Cheaper, more effective, due to previous reasons.
\item{Restitution}: Similarly, there was an important reason for creating Open Source Software. As Zigurd Magnusson states:
"We were in debt ..." : So, the company stuff considered that, as a way of restitution, they had also to provide it.  This aspect is very related to community, as commnunities in Open Source enforce to promote Open Source Software creation and restitution to the FLOSS society by companies.
\end{enumerate}
The negative thing of chosing this kind of software develpment strategy, was to know how to make money. Related to this issue, the lecture asserts: "People love our product, but dont want to pay for it".

\section{Managing people}
  Zigurd Magnusson faces an important matter related to the community building, as it is the issue of task force management. To start with, Magnusson makes a strong recommendation of Karl Fogel's "Producing Open Source Software". It is not the first time than an Open Source advocate recommends this book [2], when talking about staff or community management. From Magnusson's perspective, the book was amazing. He read it in 24 hours, discovering that he agreed 100\% on its content.\\
\\
Indeed, Fogel's book is considered to be a Bible, and should be mandatory read by people managing software producers, even more if related FLOSS software producers ~\cite{POIS00}.  In the end, producing open source software is not only centered on Open Source communities, but rather on a more wide concept around Open Source Software production, from different perspectives and on a more general view rather than what is exactly Open Source project and communities management.\\
\\
In order to manage correctly people around Open Source software, some important aspects are considered by the speaker, for instance:

\subsection{Publishing guide principles}
In order to articulate all that common facts that were true on both company staff and contributors, but did not appear in any place. This lack of procecure could drive to:
\begin{enumerate}\itemsep0pt
\item{Confuse and not differentiate roles inside the product.}
\item{Pain, due to people working on unneeded tasks.}
\end{enumerate}

\subsection{Give recognition}
From lecturer's perspective, recognition from management of the project is a very good practice, given that:
\begin{enumerate}\itemsep0pt
\item{It is important to let people know when a good job has been carried out}.
\item{It is also important to reward the best workers, but must be precise on why or why not a prize has been given to a people or group of people}.
\end{enumerate}

\subsection{Work delegation}
As the time goes by, Magnusson has found out that the responsibilities are important to be determined, but without an excessive procedure.
For this reason, he promotes the use of:
\begin{enumerate}\itemsep0pt
\item{One page overview, then pass the responsibility: As a middle point between no management and very specific management. For this reason, they create a one page overview on how the project should be like, and then pass the responsibility to what the page asserts.}
\item{For previous reason, there is a community responsible for day to day questions and particular activies around certain tasks.}
\item{The guiding principles should also fill the edges and corner cases, where the people stands.}
\end{enumerate}

\subsection{Mentorship}
\begin{enumerate}\itemsep0pt
\item{Mentors are very important for the organization, for starting people to gain knowled as fast as possible.}
\item{Mentors, however, are technical experts on one or some areas, and normally face high workloads, so it is important than they are considered to reduce their work load to help on mentorship.}
\item{It is also important to make mentors realize of their mentorship job, recognize it and help to promote it by enforcing their satisfaction due to this job. Fun, results and praise are ways of achieving so.}
\end{enumerate}
  
\subsection{Design Process}
\begin{enumerate}\itemsep0pt
\item{Design process must be flexible, and focused on fixing development issues that exist.}
\item{Designers must not be afraid of making mistakes, but must rather be prepaired in order to fix them quickly.}
\item{Clever architecting does not replace making prototypes to throw away, rather it consider this as a normal practice within development process. Software moves quickly, new ways of implementing things appear, and architects must be prepaired for it.}
\end{enumerate}

\section{Google Summer of Code}
The speaker from SilverStripe focused later on contributions from very different students provided in the Google Summer of Code. Due to this, some important contributions to the project have been provided, in different aspects such as:
\begin{itemize}\itemsep0pt
\item{Usability}
\item{Internationalization}
\item{Mashups}
\item{Database Management}
\item{Safari browser support}p
\item{Image manipulation}
\item{Reporting}
\item{etc.}
\end{itemize}
The important issue here are the lessons lernt by the speaker. Within Google Summer of Code attending, it has been checked that there are some steps to follow when hiring, as they are applied for selecting students. Among these common steps, it can be found:
\begin{enumerate}\itemsep0pt
\item{\textbf{Selection must be wise}}. It is important to provide a coding exercise to inspect how a possible next developer of the project codes, and if she/he is capable of sorting out certain issues.
\item{\textbf{Project selection}}. More focused on GSOC, with this statement Magnusson wants to highlight that projects must be selected considering how smart and simple they are taking into account the skills needed to be matched.
\item{\textbf{Start coding early}}. The programmers must have the oportunity to start coding as eary as possible, for them to demonstrate they are valid to do so.
\end{enumerate}

\section{Q\&A}
  Last, but not least, the speak ends with a Question and Answer round. In this round, some interesting questions are raised by attendants to the speak. Among them, there is a question having to do with a new movement in New Zealand around Web standards openness.\\
\\
  Sam Minée, the other speaker, clarifies how the interesting thing here is related to the fact that, being New Zealand a short, but tight, FLOSS population, when new proposals, such as previous one, appear, it is a normal thing to get people involved and interested on that particular task. So, what initially can be considered troublesome, sometimes mean benefits on the other hand.\\
\\
As final question, the speakers explain about the future of the project, the different strategies that will be followed and how the product is thought to evolve in next years.

\bibliographystyle{unsrt}
\bibliography{03-SilverStripe}{}
\end{document}
