\documentclass[11pt]{article}

\title{\textbf{CASE STUDIES II: Mercurial}}
\author{Sergio Arroutbi Braojos}
\date{\today}
\usepackage{listings}
\usepackage{cite}
\usepackage{booktabs}
\addtolength{\voffset}{-50pt}
\begin{document}

\maketitle

\section{Introduction}
On this Google Tech Talk, dated on June 2006, Brian O'Sullivan, who had been contributor to Mercurial for a while, explains the appearing of this VCS (Version Control System), who appeared in parallel to Git, in order to provide a solution to help on the fact of distributed software development. The talk is basically around two aspects:
\begin{enumerate}
\item{\textbf{Mercurial Technical Details}}. Information about Mercurial revision control system itself, and other stuff which are important from the comunity perspective.
\item{\textbf{Distributed Version Control}}. The lecturer also talks about particularities of distributed software development, and how important distributed repository control management is.
\end{enumerate}

\section{Mercurial}
From lecturer's perspective, the most important thing regarding Mercurial and any other VCS is the ease of use. He asserts:

\begin{verbatim}
"I want to focus on the issues of programming, 
and not the issues of the version control system"
\end{verbatim}

In lecturer's opinion, there are different reasons to use Mercurial. These factors are important to analyze, as can be used as a rule of thumb of those aspects which are important for distributed development comunities, and in particular for Open Source comunities:

\begin{itemize}
\item{\textbf{Straight forward to understand}}. Mercurial has a conceptual model very easy to understand. Basically, handles three concepts:
\begin{enumerate}\itemsep0pt
\item{Repository}. It is a weight-light repository, based only on a set of files. This help on the Speed when working with mercurial.
\item{Changelog}. To register changes and group information around each change.
\item{Manifest}. To register changes on each particular file for each change.
\item{File Metadata}. Which controls additional information on each file.
\end{enumerate}
The basic idea from the comunity perspective is how easy is to understand how the VCS handles changes, files and history on time, and the ease of use when working with it on a daily basis.
\item{\textbf{Easy to maintain}}. Written basically in python (95\%). There is a couple of modules also written in C. This means ease of maintaining, as Python is an easier computer programming language compared to others, and is increasing on popularity. It is, besides this, a complete VCS, ``working reasonably well'', in a code base of 12000 lines (by that date, 2006).
\item{\textbf{Fast}}. Mercurial is a very fast VCS. On a pair of very simple performance test, it seems not very much quicker compared to other systems, on not very large repositories. However, from lecturer perspective, due to its design around abstraction layers, and its strategy around optimization, Mercurial boost its competitors in terms of speed for bigger repositories. Lecturer asserts:
\begin{verbatim}
"We focus on not writing or reading more than strictly needed"
\end{verbatim}
\item{\textbf{Ease of use}}. Mercurial provide some functionalities which are straight-forward compared to other VCS. Lecturer gives an example on the ease of use of applying patches, for example. Other operation which is extremely easy is the revision navigation, which helps on aspects such as determining revisions where bugs were introduced.
\item{\textbf{Distributed}}. A very important aspect. Indeed, O'Sullivan considers that distributed VCS goes further than strictly Mercurial, and considers the concept as an important thing to handle separately. Lecturer's consideration about this topic are considered on the next chapter.
\end{itemize}

\section{Distributed VCS}
The second part of this Tech Talk is focused on the ``Distributed'' aspect of the VCS. Very related to the comunity aspect, from lecturer's perspective, \textbf{choosing the VCS has direct impact on the way that the project is going to evolve}.\\
\\
On a large company, all the people normally have access to all the parts of the project. On an Open Source project, task force is much more splitted and repository access is much more distributed. \textbf{Normally, people work with other people which talks the same language, and focus work with the same tools. This aspect is much straight-through with a distributed VCS}.\\
\\
Apart from previous aspect, distributed VCS allow developers to be off-line, meaning having no connection to central repositories. This is not possible with VCS such as Subversion or VCS. Distributed VCS, meanwhile, allow to work on a local manner. As long as the history is kept locally, and you can access your hard disk, is enough. This \textbf{distributed way of working allows more flexibility on development process}.\\
\\
Another important aspect in distributed VCS is the \textbf{ease of branching and merging}. Forking is the normal thing on distributed development, what helps on cooperation in terms of difference reconciling on the different branches which are created by the different developers. Apart from that, O'Sullivan recoginizes as well how Mercurial comunity is trying to emulate Subversion comunity example.\\
\\
Apart from that, distributed nature of Mercurial also helps on fixing the main issues that centralized version means, i.e.:
\begin{itemize}\itemsep0pt
\item{Having multiple repositories}.x
\item{Avoiding bottlenecks}.
\item{Flexibility to handle load management}.
\item{Use without network connection}.
\end{itemize}

\section{Conclusion}
Last, but not least, the lecturer's recognizes how important Subversion has been for Open Source. Indeed, O'Sullivan recoginizes the great contributions of this project and great comunity that has been built around this Open Source VCS, focusing on two aspects which he considers key in Open Source environment:
\begin{enumerate}
\item{Technical Merit}.
\item{Credit}.
\end{enumerate}
O'Sullivan highlights how both aspects are accomplished by main Subversion contributors, and how important would be for Mercurial Open Source project to start receiving contributions from them taking this considerations into account.
As conclusion, encourages the audience to use Mercurial due to the different reasons explained:
\begin{itemize}\itemsep0pt
\item{Ease of use}. Short learning curve, and similarities in certain aspects with previous VCS.
\item{Fast and efficient}. Lecturer asserts:
\begin{verbatim}
"Mercurial, in the end, is easy to use, easy to install, not too
different from CVS or Subversion, very well received from people
who has worked with it in terms of speed and eficiency
and versy easy to scale."
\end{verbatim}
\item{Ease of software maintainability}. The amount of lines of code is not much (12000 by that date). The fact of the software to be written in Python is also remarkable in terms of code maintainability.
\item{Multiplatform}. Being written in Python, allows usage of the code in different kind of Operating Systems (those where Python is available).
\end{itemize}
\bibliographystyle{unsrt}
\bibliography{05-Mercurial}{}
\end{document}
